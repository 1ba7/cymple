\section{Initial Organisation}
\subsection{Communication}
One of the first things that we did as a team was work out how we were going to communicate with each other. We chose to communicate using email, instant messaging and mobile phones, which all have different merits. Email was convenient for simultaneously informing the group of important information such meeting cancellations or major changes to the codebase. Instant messaging allowed specific group members to discuss ideas more thoroughly than possible in an email. Mobile phones allowed us to organise ad-hoc meetings inside college or find out where somebody in college. We figured we would be most effective as a group if we were able to use all of these media.

For email, we first tried setting up a Google Group, but we were unable to do this using our college email accounts. Instead, we set up an email address which forwards all incoming mail to all of our email addresses --- if we wanted to send an email to the group, we would send it to that address. For instant messaging, we simply exchanged our college email addresses and communicated using Google Talk. Some of the group set up their college Google Talk account with desktop-based instant messaging applications such as Pidgin and Trillian. For mobile phones, it was a simple case of exchanging phone numbers.

\subsection{Revision Control}
The next thing that we did was work out what revision control system we were going to use. There were many reasons for choosing Git\cite{better} --- our reasons were that Shane had experience using Git previously (and so could share his knowledge with the rest of the group easily) and the convenient web-based interface for navigating the source code and chagnes to it provided by GitHub.
