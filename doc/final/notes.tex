\section{Data Layer}
\begin{itemize}
\item Threw around some ideas --- using a red-black-tree combined with a linked list
\subsection{First Design}
\begin{itemize}
\item Binary search on sorted array\cite{Data.java}
\item To simplify the organisation of the data, used binary file format which pre-organises it
\item Description of file format, with diagram
\item Used sorted array, represented time with shorts
\item Was able to handle ~200 million entries
\item Niall's application couldn't handle this many entries, wrote my own
\item Binary search to generate graphs
\item Wrote a ruby script to generate binary file format --- rapid prototyping
\item Script is still included for reference
\item Ran into limitations with keeping the array sorted
\item Binary search was hard to get accurate results out of
\item Wasn't really designed with concurrency in mind
\end{itemize}
\subsection{Second Design}
\begin{itemize}
\item Reused header format from initial design.
\item Time normalised to 1024-values
\item Based on concept of "ListenVector"s
\item Each ListenVector is basically a 1024-size array of longs
\item This means the the application was able to support in the range of $2^{64}$ listens
\item Simpler, faster implementation
\item Has two generators --- one converts from CSV file, the other generates random data
\end{itemize}
\end{itemize}

\section{Widget toolkit}
\begin{itemize}
\item Widget toolkit is a higher-level abstraction on what is suggested in the notes
\item No static event numbers --- more flexible and maintable approach to handling events
\subsection{Features}
\begin{itemize}
\item Application, subclass of PApplet
\item Containers and auto-calculation of \textit{most} widths and heights
\item Scrollable Containers and masking
\item Javascript-like event model
\item Canvas interface-based drawing model, so widgets can be drawn to either a PGraphics or a PApplet
\item Use of pushMatrix() and popMatrix() so Widgets can freely draw from 0, 0
\end{itemize}
\subsection{Java instead of Processing}
\begin{itemize}
\item Widget toolkit implemented in Processing was intended to be used from Java
\item We found we wanted to make a more "abstract" subclass of PApplet than the one generated by Processing
\item This meant that we wanted to write Java directly rather than using that Java generated by Processing
\item This had other advantages --- ability to use features such as generics and enumerated types
\item More freedom to structure code into packages and enforce separation of the model and view/controller
\end{itemize}
\end{itemize}

\section{The Final Application}
\begin{itemize}
\item The process of bringing everything together uncovered a lot of bugs
\item Explain ListenVector stuff
\item We didn't get a chance to integrate Stiabh's graphical designs
\item Uses a `SeekCircle' for navigating by time. This idea was thought of later.
\item Displays graphs and charts at the same time, unlike previously.
\end{itemize}

\section{Conclusions}
\begin{itemize}
\item The final GUI could have been much more polished, colour scheme arbitrary.
\item A lot of people had problems using Github. This is due to Windows, not knowing CLI, and GitHub being hard to access from inside TCD. It is doable through Spoon with Socks, but Windows is bad and not everybody had a Netsoc account. Error messages which were caused by things like Windows/CRLF-style newlines vs Unix/LF newlines confused a lot of people.
\item Windows users found it difficult to get the code for the data layer working for a while because the file generation script was written in Ruby (unlike in Linux, it is not as easy as sudo apt-get install ruby in Windows).
\end{itemize}
