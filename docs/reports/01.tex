\documentclass[titlepage,a4paper]{article}

\usepackage{geometry}
\usepackage{fancyhdr}
\usepackage[
	urlbordercolor = {1 1 1},
	linkbordercolor = {1 1 1},
	citebordercolor = {1 1 1},
	urlcolor = blue,
	colorlinks = true,]{hyperref}
\usepackage{graphicx}
\setlength{\headheight}{15.2pt}
\pagestyle{fancy}
\fancyhead[L]{\textsc{1BA7 Group Project Lab 1}}
\fancyfoot{}
\fancyhead[R]{\thepage}


\begin{document}
\title{1BA7 Group Project Lab 1}
\author{Group B}
\date{\today}
\maketitle
\newcommand{\mono}[1]{\texttt{#1}}
\newcommand{\code}[1]{\texttt{#1}}

\section{Proposed Communications Mechanism}
\begin{itemize}
\item GIT for version control
\item TCD Google Talk for instant messaging
\item Google Gourp (iba7-teamb)
\end{itemize}

\section{Arragnement and Scheduling of Meeting Times}
We propose to book the study room in the Hamilton Library to hold twice-weelky
meetings totalling three hours (excluding labs and lectures). The times are as
follows:
\begin{itemize}
\item 11.00 am --- 1.00 pm Monday 
\item 12.00 pm --- 1.00 pm Tuesday
\end{itemize}

\section{Initial Work Plan}
We propose to assign jobs initially bassed on preference (i.e., initially we do
what we're most interested in and redeploy ourselves as necessary)
\begin{itemize}
\item \textbf{Maths, Data Processing}: Dan, Shane
\item \textbf{GUI}: Cris, Stuart, Padraig
\item \textbf{Project Manager}: Stiabh
\end{itemize}

\section{Vague Plans for Architecture}
Shane proposes a two-layered archicture: the lower level layer is concerned with
processes and stores the data in optimal data structures (possibly a
self-balancing binary tree combined with a doubly-linked list) and should be able
to do this independently of the GUI. The GUI will operate on top of this and query
the layer. Various ideas have been suggested for the GUI although these have not
been compiled and made consistent with each other yet.
We plan to do a mockup on paper that will incorporate the ideas we have for the GUI.
\par Some of the ideas we've had include a peak/trough graph with a mousewheel as 
zoom tool, date filters, use of labels for `new entries' and use of colour gradients.
There have been a lot of cool visualations based on last.fm data --- some influence
could be drawn from these.

\section{Preliminary Deadlines}
\begin{itemize}
\item Wednesday (April 1st, seriously): Git tutorial: Shane will teach the group how to use setup and use git
\item Thursday (April 2nd): Presentation for mock up of the GUI and data structures
\end{itemize}

\end{document}
