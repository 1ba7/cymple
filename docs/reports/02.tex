\documentclass[titlepage,a4paper]{article}

\usepackage{geometry}
\usepackage{fancyhdr}
\usepackage[
	urlbordercolor = {1 1 1},
	linkbordercolor = {1 1 1},
	citebordercolor = {1 1 1},
	urlcolor = blue,
	colorlinks = true,]{hyperref}
\usepackage{graphicx}
\setlength{\headheight}{15.2pt}
\pagestyle{fancy}
\fancyhead[L]{\textsc{1BA7 Group Project Lab 2}}
\fancyfoot{}
\fancyhead[R]{\thepage}

\begin{document}
\title{1BA7 Group Project Lab 2}
\author{Group B}
\date{\today}
\maketitle
\newcommand{\mono}[1]{\texttt{#1}}
\newcommand{\code}[1]{\texttt{#1}}

\section{Progress Overview}
Here is a summary of the progress we've made over the last week.
\begin{itemize}
\item \mono{git} --- we've all set up git and are getting comfortable using it.
\item Design --- we've designed, on paper a mockup for the GUI, discussed it and
refined the design.
\item Arhicecture Overview --- we've figured out how we're going to partition the
code (detailed below)
\item Roadmap --- we've outlined a roadmap of what we're going to do
\end{itemize}

\section{Source Code Management}
We've decided to use \code{git} for source code management in favour of
\code{svn}. We're going to use GitHub to host our code. It provides a really
nice web interface that lists all the files, branches and commit data. It makes
it very easy to track who's made what changes to what files, etc.
\begin{itemize}
\item We've all set up accounts on \href{http://github.com}{GitHub}, and our
centralised repository is hosted \href{http://github.com/1ba7/}{here}.
\item Everybody has installed \code{msysgit} (Windows port of \code{git}) on
their computers. Shane ran a tutorial explaining how to use it and distributed
documentation to everybody.
\item There have been some issues with getting \code{msysgit} running on the
lab machines in college, but once these get sorted out we should all be able to
start making commits to the repository.
\item Our GitHub repository is the first result when you Google for "1ba7".
\end{itemize}

\section{Graphical Design}
\begin{itemize}
\item Last week, Cris did an initial design on paper of what the GUI should look
like.
\item This was discussed at a meeting during the week.
\item We had some more ideas and refinements and we redid the GUI (on paper).
\item We're happy enough with the current design to work with it for now.
\item Cris and Stiabh are working on doing digitising these mockups.
\end{itemize}

\section{Architecture Overview}
Initially we decided upon a two-layered approach: the GUI and the data
backend. After thinking about this a little more it was decided that we actually
want three layers.
\begin{itemize}
\item data layer --- this provides a well-defined API for the application to
use to get useful data from the \mono{.csv} file. It handles all the parsing of
the \mono{.csv} file and stores the data in an efficient way.
\item widget toolkit --- this provides a set of high-level, reusable widgets
(implemented in Processing) which are used in the application.
\item application --- this uses the data layer and the widget toolkit to
implement an application which visualises the data from the \code{.csv} file.
\end{itemize}
\par Shane has started work on documenting all the classes that these APIs will
need to provide in YAML. These YAML files will be used to create .java and .pde
stub files.

\section{Roadmap}
Here is our Roadmap as outlined in an email sent to the mailing list by Shane:
\begin{itemize}
\item Git ---
We need to make sure that we're all comfortable with using git and
github. There are a few things I didn't cover properly in the git
tutorial and Stiabh missed it completely. My messages about github
probably caused confusion more than anything else, so I think we
should have another session where everybody plays around with git and
makes a few commits to our github just to make sure they're
comfortable with it. I will also explain SSH keys properly and help
you get set up so that you can make commits to github from both your
laptop and the machines on campus.
See: \url{http://en.wikipedia.org/wiki/Git_(software)}
See: \url{http://nathanj.github.com/gitguide/}
See: \url{http://www-cs-students.stanford.edu/~blynn/gitmagic/} (if you feel
like learning how to use git from the command-line)
Our github page: \url{http://github.com/1ba7}

\item Finalise design of the application ---
Once everybody's comfortable with git, the next thing we should focus
on is the design of the application. This means we need to figure out
in a concrete way what kind of sets of data we want our application to
be capable of visualising and what the user interface for this
application is to look like (in terms of buttons, scrollbars, etc). We
definitely shouldn't rush this, but we do need to get it done
reasonably soon.

\item Design APIs for the data layer and the toolkit
This step absolutely depends on the design of the application being
finalised if it is to be done effectively. We need to get a concrete
idea of what classes and methods the application is going to need
access to in the data layers and what widgets and events we need to
provide the application with in the widget toolkit
See: \url{http://en.wikipedia.org/wiki/Encapsulation_(computer_science)}
See: \url{http://en.wikipedia.org/wiki/Widget_toolkit}
See: \url{http://java.sun.com/docs/books/tutorial/java/concepts/inheritance.html}

\item Implement the toolkit and the data layer
This is probably going to take the longest out of any of the steps,
but thankfully, the toolkit and the data layer can be worked on in
parallel because they will be independent of each other. As in, it
won't be the case that the toolkit can't be worked on because it
depends on a certain feature in the data layer because they are
independent. We should split into two teams, one working on each of
these layers. Although we will be mainly focused onthe layer that
we're working on, we should still be familiar with the codebase of the
layer that we are not ourselves working on.

\item Bringing it all together
Once we have the toolkit and the data layer done, we can then put them
together and make the actual application. Unfortunately, this is
probably going to unearth a number of unexpected bugs, so we should
allocate sufficient time for the completion of this task.

\end{itemize}
\end{document}
