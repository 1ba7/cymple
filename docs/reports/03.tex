\documentclass[titlepage,a4paper]{article}

\usepackage{geometry}
\usepackage{fancyhdr}
\usepackage[
	urlbordercolor = {1 1 1},
	linkbordercolor = {1 1 1},
	citebordercolor = {1 1 1},
	urlcolor = blue,
	colorlinks = true,]{hyperref}
\usepackage{graphicx}
\setlength{\headheight}{15.2pt}
\pagestyle{fancy}
\fancyhead[L]{\textsc{1BA7 Group Project Lab 3}}
\fancyfoot{}
\fancyhead[R]{\thepage}

\begin{document}
\title{1BA7 Group Project Lab 3}
\author{Group B}
\date{\today}
\maketitle
\newcommand{\mono}[1]{\texttt{#1}}
\newcommand{\code}[1]{\texttt{#1}}

\section{Progress Overview}
Having Easter Monday off meant that we didn't have our usual meeting slot on
a Monday, so we haven't made as much progress in the last week as we would have
liked to.
\begin{itemize}
\item Architecture for Data Layer --- we've designed the arhictecture for the
data layer properly. All classes, methods and attributes have been documented
thoroughly. Shane is writing a Ruby script that will convert these
specifications into \code{.java} and \code{.pde} stubs.
\item Stiabh been working on the icons for use in the application.
\item The others have been working on implementing the mockup in processing
using a the primitive Widget toolkit outlined in the lecture notes.
\end{itemize}

\section{Architecture for Data Layer}
Keeping in mind that we are liable to run out of memory when given a large
set of entries, and disregarding the principle of "Premature optimisation is
the root of all evil", we've designed a set of classes and interfaces designed 
to keep as little data in memory as possible. Features of this architecture
include:
\begin{itemize}
\item Small \code{Listen} class --- The \mono{Listen} class used in our
application, which links a user, track and a time, takes up only 16 bytes. An
empty Java \mono{Object} takes up 8 bytes, so this is about as small it could
possibly be. 
\item The Artist, Album, Track and User classes have mechanisms in place to
avoid duplicate objects being instanciated.
\item Support for storing the data from the CSV file in a binary file format.
This avoids having to store everything in memory, at the cost of speed. This has
not yet been implemented and may not be implemented, but the classes are designed
in such a way that it is a possibility if necessary.
\end{itemize}

\section{Visual Theme}
Stiabh has been working on the cosmetic aspects of the application ---
making icons, animations, etc. He's been doing this in 3DS Max 9 and will
eventually export what he's been doing as \mono{.gif} and \mono{.png} files.

\section{Digitising the mockup}
Cris, Stuart and Padraig have been working on creating a \mono{processing}
implementation of the GUI that was designed on paper. They have so far been
using the primitive \mono{Widget} toolkit outlined in the lecture notes.
At the same time, Shane has been designing a more flexible, reusable,
object-oriented toolkit for use the final application. What will likely happen
is that the toolkit which people are right now will evolve over time into the
one which Shane is designing.
\end{document}
